\input{./econtexPathsOnly.tex}\documentclass{beamer}
\usepackage{cancel}
\usepackage{CDCShortcuts}
\renewcommand{\GDPLev}{\ensuremath{\pmb{Y}}}
\renewcommand{\gdpLev}{\ensuremath{\pmb{y}}}
\renewcommand{\ptyLev}{\ensuremath{z}}


\usepackage{wasysym}
\providecommand*{\newblock}{}
\usepackage{natbib}

\mode<presentation>
{
  \usetheme{Warsaw}
  % or ...
  \setbeamercovered{transparent}
}

% 
\beamerdefaultoverlayspecification{<+->}

%\setbeamertemplate{navigation symbols}{}  % Take away navigation symbols

\usetheme{Warsaw}

\setbeamersize{text margin left=3mm}
\setbeamersize{text margin right=3mm}


%_____________ Opening slide _______________________

\title[A Tractable Model]{A Tractable Model of Precautionary Reserves, \\ Net Foreign Assets, or Sovereign Wealth Funds}
\author[Carroll, Jeanne]{Christopher Carroll and Olivier Jeanne}
\institute[JHU]{Johns Hopkins University}
\date[\today]{\today}

\begin{document}

\begin{frame}[plain]
  \titlepage
\end{frame}


%_____________ 1st section  ____________
\section{Introduction}
\subsection{Motivation}

\begin{frame}
\frametitle{Motivation}

Three Hot Topics In International Macro: \pause
\begin{itemize}
\item Huge Reserve Accumulation By Fast-Growing Developing Economies
\begin{itemize}
\item China
\end{itemize}
\item Surprising ``Upstream'' Capital Flows: Developing $\rightarrow$ Rich Countries
\begin{itemize}
\item China -- Following Japan, Korea, Taiwan, Singapore, Hong Kong, ...
\end{itemize}
\item Sovereign Wealth Funds
\begin{itemize}
\item Mainly Oil-Rich Countries
\end{itemize}
\end{itemize}

\end{frame}
\begin{frame}
\frametitle{Connection?}

\pause Precautionary Motives Commonly Cited In All Three Cases

\pause \begin{itemize}
    \item Our Model of Precautionary Net Foreign Assets:
        \begin{itemize}
            \item Tractable. \pause Tractable! \pause \pause TRACTABLE!!!
\pause      \item The Natural Extension of the Ramsey Model
\pause      \item Shows Eqbm Relation Between Precautionary, Other Motives
        \end{itemize}
    \item Two applications
        \begin{itemize}
            \item Economic Growth and Capital Flows
            \item Impact of Reducing Global Financial Imbalances
        \end{itemize}
    \end{itemize}
\end{frame}

\subsection{Literature}
\begin{frame}
\frametitle{Literature}
    \begin{itemize}
    \item Aggregated Micro Model
\begin{itemize}
\item ``Real'' microfoundations!
\end{itemize}
    \item Builds on \cite{toche:urisk}
    \item Related: \cite{fogliPerriMod}, \cite{mqrImbal}, \cite{sandriGrowth}
    \item Other Approaches: \cite{cfg:globimbalances}
    \end{itemize}

\end{frame}

\subsection{Structure}
\begin{frame}
\frametitle{Structure}
    \begin{itemize}
    \item Model
    \item Calibration and Simulation
    \item Applications
\begin{itemize}
    \item Growth and Capital Flows
    \item Complete World Knowledge (General Equilibrium)
\end{itemize}
    \end{itemize}
\end{frame}

%----------------------------------------------------------------------------------------------------------------------------------
\subsection{Overview}
\begin{frame}
\frametitle{Overview}
    \begin{itemize}
    \item Small Open Economy
    \item Balanced Growth Path With Population And Productivity Growth
    \item Accumulate Buffer Stock to Self-Insure Against Unemployment 
    \item NFA: Aggregate Stock of Wealth Minus Domestic Capital Stock
    \item Closed-Form Solutions For Equilibrium
    \end{itemize}

\end{frame}

\section{Model}
\subsection{Macroeconomy}

\begin{frame}
\frametitle{Macroeconomic Assumptions}
    \begin{itemize}
    \item Domestic output is produced with the Cobb-Douglas function:
\begin{eqnarray}
\GDPLev_t=\KLev_t^{\kapShare}(\ptyLev_t \LLev_t)^{1-\kapShare},
\end{eqnarray}
    \item Labor productivity increases by $\WGro$ in every period,
\begin{eqnarray}
\ptyLev_{t+1}=\WGro \ptyLev_t.
\end{eqnarray}
    \item Capital perfectly mobile internationally,
\begin{eqnarray}
\overbrace{\DeprFac}^{\equiv 1-\delta}+\kapShare \frac{\GDPLev_t}{\KLev_t}=\Rfree,
\end{eqnarray}
    \item Capital-to-output ratio is constant and equal to,
\begin{eqnarray}
\frac{\KLev}{\GDPLev}=\frac{\kapShare}{\Rfree-\DeprFac}.
\end{eqnarray}
    \end{itemize}
\end{frame}

%----------------------------------------------------------------------------------------------------------------------------------
\subsection{People}

\begin{frame}
\frametitle{People and Populations}
    \begin{itemize}
    \item Each worker is part of a single `generation' born at the same time
    \item Size of generation born at $t: \PopGro^t$.
    \item Life Stages:
\begin{itemize}
\item Employment
\item Unemployment/Retirement
\item Death
\end{itemize}
    \item Transitions to unemployment and death are Poisson processes
\begin{itemize}
\item Flow probabilities $\urate$ and $\PDies$.
\end{itemize}
    \item Employed and Unemployed Populations:
\input ../equations/Pop.tex
\end{itemize}
\end{frame}


%----------------------------------------------------------------------------------------------------------------------------------
\subsection{Balanced Growth}
\begin{frame}
\frametitle{Balanced Growth}
    \begin{itemize}
    \item Capital and output grow at constant rates
%\begin{eqnarray}
%\frac{\KLev_{t+1}}{\KLev_t}=\frac{\GDPLev_{t+1}}{\GDPLev_t}=\PopGro \WGro.
%\end{eqnarray}

    \item Real wage grows by factor $\WGro$ in every period.

    \item Main variable of interest= $\NFALev_t$, the aggregate net foreign assets of the economy at the beginning of period $t$.
\begin{eqnarray}
\NFALev_t = \BLev_t - \KLev_t.
\end{eqnarray}
    \end{itemize}
\end{frame}



%----------------------------------------------------------------------------------------------------------------------------------
\begin{frame}
\frametitle{The microeconomic consumer's problem}
    \begin{itemize}
    \item Budget constraint of individual:
\input ../equations/ibc.tex
    \item Worker's labor supply $\labor$ grows by a factor $\XperGro$ per period over his lifetime,
\input ../equations/IndLabSup.tex
    \item For consumer who remains employed, labor income grows by 
\input ../equations/XperGro.tex
    \end{itemize}
\end{frame}

%-------------------------------------------------------------------------------
\subsection{The Microeconomic Problem}
\begin{frame}
\frametitle{The microeconomic consumer's problem}
    \begin{itemize}
    \item Unemployment: Complete and permanent destruction of $h$

    \item CRRA felicity $\uFunc(\bullet)=
\bullet^{1-\CRRA}/(1-\CRRA)$; geometric discounting at $\beta$ 

    \item Unemployed convert their wealth into annuities.

    \item Solution to the unemployed consumer's optimization problem,
\begin{eqnarray*}
\cLevU_{t} & = & \MPCU \bLev_t,
\end{eqnarray*}
where $\MPC$ is the marginal propensity to consume,
\begin{eqnarray*}
\MPCU & \equiv & 1- \PLives\frac{(\beta \Rfree)^{1/\CRRA}}{\Rfree}.
\end{eqnarray*}

    
    \end{itemize}

\end{frame}

%----------------------------------------------------------------------------------------------------------------------------------
\begin{frame}
\frametitle{The microeconomic consumer's problem}
    \begin{itemize}
    \item `Growth impatience condition':
\begin{eqnarray*}
\PatPGro & \equiv & \frac{(\beta \Rfree)^{1/\CRRA}}{\PGro} <1
\end{eqnarray*}
necessary for finite target ratio of wealth to income (\cite{BufferStockTheory})
    \item Defining nonbold variables as, e.g., $\cRatE_{t}=\cLevE_{t}/(\Wage_{t}\labor_{t})$, we get
\input ../equations/ibcnorm.tex
\input ../equations/cetp1.tex

    \item Saddle-point stable dynamics. 
    \end{itemize}

\end{frame}

%----------------------------------------------------------------------------------------------------------------------------------
\begin{frame}
\frametitle{Phase Diagram}
    \begin{figure}
    \centering
    \includegraphics[width=.65\textwidth]{../figures/phaseDiag.pdf}
    \end{figure}
\end{frame}

%----------------------------------------------------------------------------------------------------------------------------------

\begin{frame}
\frametitle{The Growth Impatience Condition}
\begin{itemize}
    \item Target wealth-to-income ratio: impatience vs prudence.

    \item Closed-form solution for the target wealth-to-income ratio
\input ../equations/bTargE.tex
    \item \begin{eqnarray}
\frac{\partial \bTarg}{\partial \urate}>0,
\frac{\partial \bTarg}{\partial \beta}>0,
\frac{\partial \bTarg}{\partial \PGro}<0.
\label{eq:compstat}
\end{eqnarray}


\begin{eqnarray}
\frac{\partial \bTarg}{\partial \CRRA}>0.
\label{eq:db/drho=}
\end{eqnarray}
    \item The response of $\bTarg$ to $\Rfree$ is ambiguous. 
\end{itemize}
\end{frame}



%----------------------------------------------------------------------------------------------------------------------------------
\subsection{Foreign Assets}
\begin{frame}
\frametitle{Foreign Assets}
    \begin{itemize}
    \item Ratio of employed workers' wealth to output,
\input ../equations/BRatEIndAgg.tex
where $\LGro$ is the factor by which the share of a generation in total labor supply shrinks every period.

    \item The Level of Unemployed Workers' Wealth is 
\input ../equations/BLevUtp1.tex
\end{itemize}
\end{frame}

\begin{frame}\frametitle{Foreign Assets (cont)}
\begin{itemize}
\item Steady state ratio of net foreign assets to GDP
    \input ../equations/NFARat
\item Depends on Employed Workers' Target Savings
\end{itemize}
\end{frame}

\begin{frame}\frametitle{`Stakes'}
\begin{itemize}
    \item Model with no stakes
\input ../equations/BRatENostake
    \item Model with stakes yielding a representative agent
\input ../equations/BRatEStakes
\end{itemize}
where
\input ../equations/bTargEStakes

\end{frame}

\begin{frame}\frametitle{Advantages Of Model With Stakes}

\begin{itemize}
   \item Closed-form solution for steady state
   \item Simple to characterize transition dynamics
\end{itemize}

\end{frame}



%----------------------------------------------------------------------------------------------------------------------------------
\section{Calibration And Simulation}
\subsection{Parameter Values}
\begin{frame}
\frametitle{Calibration and Simulation}


\centerline{\bf Table 1}

\begin{center}
\begin{tabular}{|c|c|c|c|c|c|c|c|c|c|}
  \hline
  % after \\: \hline or \cline{col1-col2} \cline{col3-col4} ...
  $\kapShare$ & $\delta$ & $\PopGro$ & $\WGro$ & $\Rfree$ & $\beta^{-1}$ & $\PtyGro$ & $\urate$ & $\CRRA$ & $\pDies$ \\ \hline
  0.3 & 0.06 & 1.01 & 1.04 & 1.04 & 1.04 & 1.01 & 0.025 & 2 & 0.05 \\
  \hline
\end{tabular}
\end{center}

    \begin{itemize}
    \item ${\NFALev}/{\GDPLev}=0.17$ in the model with no stakes
    \item ${\NFALev}/{\GDPLev}=0.79$ in the model with stakes
    \end{itemize}

\end{frame}

\subsection{Paths}
%----------------------------------------------------------------------------------------------------------------------------------
\begin{frame}
\frametitle{Paths}
    \begin{figure}
    \centering
    \includegraphics[width=.6\textwidth]{../figures/paths.pdf}
    \end{figure}
  \end{frame}

%----------------------------------------------------------------------------------------------------------------------------------
\subsection{Sensitivity Analysis}

\begin{frame}
\frametitle{Sensitivity analysis}
    \begin{figure}
    \centering
    \includegraphics[width=.6\textwidth]{../figures/sensitivity.pdf}
    \end{figure}
\end{frame}

%----------------------------------------------------------------------------------------------------------------------------------

\subsection{Social Insurance}
\begin{frame}
\frametitle{Social Insurance}
    \begin{itemize}
    \item Many countries have social transfers to unemployed/retired
    \item New assumption: labor income tax on the employed in order to finance transfers to
the unemployed.
    \item Unemployed receive transfer whose value is a multiple $\Severance$ of the labor income that they would have received if they had remained employed.
    \item New formula for target wealth-to-income ratio.  Going through the same steps as before, we get
\input ../equations/bTargWithSocIns
%
%\begin{eqnarray}
%\bTarg(\Severance) = \bTarg 
%\left[1-\Severance \left(\frac{\urate \PtyGro}{\PopGro}+\MPC %\left(1+\frac{\PatPGro^{-\CRRA}-1}{\urate}\right)^{1/\CRRA}\right)\right],
%\end{eqnarray}
    \end{itemize}

\end{frame}

%----------------------------------------------------------------------------------------------------------------------------------

\begin{frame}
\frametitle{Social insurance}
    \begin{figure}
    \centering
    \includegraphics[width=.6\textwidth]{../figures/socins.pdf}
    \end{figure}
\end{frame}

%----------------------------------------------------------------------------------------------------------------------------------

\section{Applications}
\subsection{Growth And Saving}
\begin{frame}
\frametitle{Growth And Saving}
    \begin{itemize}
    \item Theory: Good Growth Prospects $\rightarrow$ Should Borrow to Invest
    \item Data: Fast-Growing Countries {\it Export} Capital 
\begin{itemize}
\item \cite{carroll&weil:crcs}; \cite{lss:whatdrives}; \cite{aps:sgi}; Gourinchas and Jeanne, 2007, Prasad, Rajan and Subramanian (2007); Sandri (2008)
\end{itemize}
    \item Can this model shed light on this puzzle?
    \item Yes, if growth take-off entails idiosyncratic risk (both $\WGro$ and $\urate$ go up).
    \end{itemize}

\end{frame}

%----------------------------------------------------------------------------------------------------------------------------------

\begin{frame}
\frametitle{Growth and capital flows}
    \begin{figure}
    \centering
    \includegraphics[width=.6\textwidth]{../figures/capOutflows.pdf}
    \end{figure}
\end{frame}

%----------------------------------------------------------------------------------------------------------------------------------
\subsection{Resorbing Global Imbalances}
\begin{frame}
\frametitle{World General Equilibrium}
    \begin{itemize}
    \item Small economy assumption not appropriate to study global savings glut or adjustment of global financial imbalances. 
    \item Study steady state equilibria in two-country extension of the model.
    \item Global interest rate $\Rfree$ endogenous
\begin{eqnarray}
\NFALev_h+\NFALev_f=0,
\end{eqnarray}
    \end{itemize}

\end{frame}

%----------------------------------------------------------------------------------------------------------------------------------
\begin{frame}
\frametitle{General Equilibrium}
    \begin{itemize}
    \item Two countries identical except for size (h=20\%, f=80\%) and level of social insurance ($\Severance_h=1.5$, $\Severance_f=0.75$).
    \item This implies
    \begin{eqnarray}
    \frac{\NFALev_h}{\GDPLev_h}&=&-0.5 \\
    \frac{\NFALev_f}{\GDPLev_f}&=&0.125
    \end{eqnarray}
    \item What is impact of increasing foreign social insurance to the home level?
    \end{itemize}
\end{frame}

%----------------------------------------------------------------------------------------------------------------------------------

\begin{frame}
\frametitle{General equilibrium}
    \begin{figure}
    \centering
    \includegraphics[width=.55\textwidth]{../figures/geneqbm.pdf}
    \end{figure}
\end{frame}

%----------------------------------------------------------------------------------------------------------------------------------
\section{Conclusions}

\begin{frame}
\frametitle{Conclusions}
    \begin{itemize}
    \item Tractable model of net foreign assets of small open economy
    \item Two applications
        \begin{itemize}
        \item Relationship between growth and capital flows
        \item Long-run implications of reducing global imbalances.
        \end{itemize}
    \item Extensions for future research: portfolio choice, real exchange rates, asset prices, etc.
    \end{itemize}

\end{frame}

\tiny 
\beamerdefaultoverlayspecification{<*>}

\begin{frame}[allowframebreaks]
\frametitle{\textbf{References}}
\tiny
\input bibMake
\end{frame}

\end{document} 
