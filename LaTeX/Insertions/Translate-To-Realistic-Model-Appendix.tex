\documentclass[../../cjSOE.tex]{subfiles}
%\documentclass{article}
% \usepackage{url}
% \usepackage{natbib}
% \usepackage{csvsimple}
% \usepackage{graphicx}
% \usepackage[flushleft]{threeparttable}
% \usepackage{color}
% \usepackage{amsmath,amssymb}
% \usepackage{moreverb}
% \usepackage{caption}
% \usepackage{multirow}

\begin{document}
%\begin{verbatimwrite}{../../insert_saving-and-uncert_2}
\subsection{Model with Time Preference Heterogeneity}

In this model, the economy consists of a continuum of households of mass one distributed on the unit interval. Households die with a constant probability $\mathrm{D}=1-\PLives$ between periods. (This is different from our baseline model in which households only face probability of dying after they become unemployed.) The income process was described in section \ref{subsec:robustness}. Each household maximizes expected discounted utility from consumption:
\begin{equation}
\max~~\mathbb{E}_{t}\sum_{n=0}^{\infty}(\PLives\beta)^{n}\util(\cFunc_{t+n})
\end{equation}
The household consumption function $\cFunc$ satisfies:
\begin{eqnarray*}
v(m_{t}) & = & \max_{c_{t}} \util(\cFunc(m_{t}))+\beta\PLives\mathbb{E}_{t}\permShk_{t+1}^{1-\CRRA}v(m_{t+1}), \label{eq:tran}
\\ & \text{s.t.} & \nonumber
\\a_{t} & = & m_{t}-\cFunc(m_{t})
\\k_{t+1} & = & \frac{a_{t}}{\PLives \permShk_{t+1}}
\\m_{t+1} & = & (\DeprFac+r_{t})k_{t+1}+\tshk_{t+1}
\\a_{t} & \geq & 0
\end{eqnarray*}
where the variables are divided by the level of permanent income $\pLev = p_{t}\pmb{W}$, so that
when aggregate shocks are shut down, the only state variable is (normalized) cash-on hand $m_{t}$.
The production function is Cobb-Douglas:
\begin{align}
ZK^{\alpha}(\ell L)^{1-\alpha}
\end{align}
The aggregate wage rate $\pmb{W}_{t}$ is determined by the aggregate productivity $Z_{t}$, capital stock $K_{t}$, and the aggregate supply of labor $L_{t}$:
\begin{align}
\pmb{W}_{t}=(1-\alpha)Z_{t}(\frac{K_{t}}{\ell L})^{\alpha}
\end{align}
$L_{t}$ is driven by two aggregate shocks:
\begin{align}
L_{t}=P_{t}\Theta_{t}
\\P_{t}=P_{t-1}\Psi_{t}
\end{align}
where $P_{t}$ is aggregate permanent productivity, 	$\Psi_{t}$ is the aggregate permanent shock
and $\Theta_{t}$ is the aggregate transitory shock.\footnote{Note that $\Psi$ is the capitalized version of the Greek letter $\psi$ used for the idiosyncratic permanent shock; similarly $\Theta$ is the capitalized $\theta$}

%\end{verbatimwrite}
%\subsection{Model with Time Preference Heterogeneity}

In this model, the economy consists of a continuum of households of mass one distributed on the unit interval. Households die with a constant probability $D=1-\PLives$ between periods. This is different from the baseline model in which households only face probability of dying after they become unemployed. The income process of a household has been described in section 3.2. Each household maximizes expected discount utility from consumption:
\begin{eqnarray}
\max\mathbb{E}_{t}\sum_{n=0}^{\infty}(\PLives\beta)^{n}\util(\cFunc_{t+n})
\end{eqnarray}
The household consumption functions satisfies:
\begin{eqnarray}
v(m_{t}) & = & \max_{c_{t}} \util(\cFunc_{t}(m_{t}))+\beta\PLives\mathbb{E}_{t}(\permShk_{t+1})^{1-\rho}v(m_{t+1}), \label{eq:tran}
\\ \text{s.t.}, \nonumber
\\a_{t} & = & m_{t}-\cFunc(m_{t})
\\k_{t+1} & = & \frac{a_{t}}{\PLives \permShk_{t+1}}
\\m_{t+1} & = & (\DeprFac+r_{t})k_{t+1}+\tshk_{t+1}
\\a_{t} & \geq & 0
\end{eqnarray}
where the variables are divided by the level of permanent income $\pLev = p_{t}\pmb{W}$, so that
when aggregate shocks are shut down, the only state variable is (normalized) cash-on hand $m_{t}$.
The production function is Cobb-Douglass:
\begin{align}
ZK^{\alpha}(\ell L)^{1-\alpha}
\end{align}
The aggregate wage rate $\pmb{W}_{t}$ is determined by the aggregate productivity $Z_{t}$, capital stock $K_{t}$, and the aggregate supply of labor $L_{t}$:
\begin{align}
\pmb{W}_{t}=(1-\alpha)Z_{t}(\frac{K_{t}}{\ell L})^{\alpha}
\end{align}
$L_{t}$ is driven by two aggregate shocks:
\begin{align}
L_{t}=P_{t}\Theta_{t}
\\P_{t}=P_{t-1}\Psi_{t}
\end{align}
where $P_{t}$ is aggregate permanent productivity, 	$\Psi_{t}$ is the aggregate permanent shock
and $\Theta_{t}$ is the aggregate transitory shock.\footnote{Note that $\Psi$ is the capitalized version of the Greek letter $\psi$ used for the idiosyncratic permanent shock; similarly $\Theta$ is the capitalized $\theta$}



%\bibliographystyle{ier}\bibliography{Reference}
\end{document}

